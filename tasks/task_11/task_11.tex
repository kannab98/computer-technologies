\input{text/pre}

\NewDocumentCommand{\codeword}{v}{%
\texttt{\textcolor{gray}{#1}}%
}
\input{python.tex}

\title{Компьютерные технологии}
\author{Карусевич А. А.}
\date{}



\begin{document}
\maketitle

\paragraph{Задание 11} Создайте модель процесса остывания стеклянного стакана с горячим кофе при 
комнатных условиях. Постройте графики изменения температуры с учетом теплопроводности, 
конвекции и испарения, а также оцените точность интегрирования в зависимости от схемы 
интегрирования и величины шага интегрирования.


\section{Алгоритм решения}
Для моделирования процесса остывания стакана с кофе необходимо учесть несколько эффектов. Во-первых, теплопроводность. 

\paragraph{Учет теплопроводности} Теплопроводность описывается уравнением
теплопроводности (Heat Equation) и в общем виде выглядит следующим образом:
\begin{equation}
	\frac{\partial u}{\partial t} = a^2 \Delta u + f(r,t),
\end{equation}
где $u(x,y,t)$ - температура в пространстве и времени, $a^2$ —  коэффициентом температуропроводности, $\Delta$ — оператор Лапласа и $f(r,t)$ — функция тепловых источников. 
В нашем случае кофе наливается горячим, а после остывает при комнатных
температурах, где источники отсутствуют, поэтому $f = 0$.

Чтобы описать систему кофе-стакан-воздух, введем зависимость коэффициента температуропроводности $a^2$ от координат. Соответственно в тех местах,
где расположен кофе, будет значение температуропроводности кофе, где стекло - там температуропроводность стекла и т.д.


Рассматривать задачу будем в двух измерениях X, Y, поскольку при использовании цилиндрической системы координат от
угла ничего не будет зависеть.
В двумерной декартовой системе координат оператор Лапласа запишется как
\begin{equation}
	\Delta u = \frac{\partial^2 u}{\partial x^2} + \frac{\partial^2 u}{\partial
    y^2}.
\end{equation}
Для численного расчета Лапласиана будем применять метод конечных разностей, например для частной производной по $x$:

\begin{equation}
	\frac{\partial^2 u(x_i,t)}{\partial x^2} = \frac{u(x_{i+1},t) -2 u(x_{i},t) + u(x_{i-1},t)}{\Delta x^2},
\end{equation}
где $x_i$ - принадлежит предварительно распределенной координатной сетке с заданным шагом.

\paragraph{Учет конвекции} Чтобы учесть конвекционную передачу тепла от жидкости стенке, воспользуемся законом 
Ньютона — Рихмана, описывающего теплопередачу от жидкости телу. Запишем в следующей форме:
\begin{equation}
	\frac{\partial u}{\partial n} = \frac{\alpha}{\lambda}(u_s - u),
\end{equation}
где $\frac{\partial u}{\partial n}$ - производная по нормали к поверхности тела на границе тело-жидкость, т.е. стенки стакана, $u_s$ - 
температура поверхности, $\alpha$ - коэффициент теплоотдачи, $\lambda$ - коэффициент теплопроводности.

В таком виде данный закон выступает граничным условием третьего рода для уравнения теплопроводности, описанного выше.

\paragraph{Учет испарения} В общем случае учет испарения является достаточно трудоемкой задачей. В данной работе охлаждение жидкости засчет
испарения будет учитываться в зависимости от скорости испарения массы жидкости, а также удельной теплоты испарения, откуда будет находится изменение температуры.

Скорость испарения $W$ (кг/ч) описывается как
\begin{equation}
	W = S (b + 0.0174 \cdot V) P_V(T) (1 - \frac{h}{100}),
\end{equation}
где $S$ - площадь испаряемой жидкости (м$^2$), $b$ - фактор скорости подвижности окружающего воздуха, $V$ - скорость воздуха на поверхности испаряемой жидкости,
$P_V(T)$ - давление насыщенного пара, зависящее от температуры, $h$ - влажность воздуха в процентах.

Зная массу испаряемой жидкости, а также удельную теплоту испарения $L \simeq 2260 $ кДж/кг, можно найти теплоту, потраченную жидкостью на испарение,
а значит и падение температуры в верхнем слое.


\section{Результаты моделирования}
Моделирование системы производится на языке Python.
Используется библиотека SciPy, а также встроенный для решения дифференциальных уравнений метод \text{integrate.solve\_ivp}.
В данном методе используется алгоритм Рунге-Кутты 5-го порядка, а шаг интегрирования выбирается так,
чтобы ошибка не превышала наперед заданного значения относительной $\varepsilon_r \leq 10^{-3}$ и абсолютной
ошибок $\varepsilon_a \leq 10^{-6}$ интегрирования.
Исходный код программы приведен в листинге \ref{lst:task11}.

Результаты моделирования приведены на рисунках
\ref{fig:init}, \ref{fig:10}, \ref{fig:30}, \ref{fig:120}, \ref{fig:600}, \ref{fig:temp}.

Кофе имеет изначальную температуру 350 К, а стекло и воздух вокруг 300 К.
Изначальное распределение температуры приведено на рис. \ref{fig:init}. На рисунках \ref{fig:10}, \ref{fig:30}, \ref{fig:120}, \ref{fig:600} показаны распределения 
температуры в разные моменты времени - 10, 30, 120 секунд с начала моделирования

На рис. \ref{fig:temp} приведена зависимость температуры верхней части кофе от времени.

\begin{figure}[H]
	\center
	\includegraphics[width=.6\linewidth]{imgs_11/init.png}
	\caption{Начальное распределение температуры. До $x=4$ см расположено кофе, далее до $x=5$ см стеклянный стакан.
	Далее пространство заполнено воздухом}
	\label{fig:init}
\end{figure}
\begin{figure}[H]
	\center
	\includegraphics[width=.6\linewidth]{imgs_11/10s.png}
	\caption{Распределение температуры в $t=10$с}
	\label{fig:10}
\end{figure}
\begin{figure}[H]
	\center
	\includegraphics[width=.6\linewidth]{imgs_11/30s.png}
	\caption{Распределение температуры в $t=30$с}
	\label{fig:30}
\end{figure}
\begin{figure}[H]
	\center
	\includegraphics[width=.6\linewidth]{imgs_11/120s.png}
	\caption{Распределение температуры в $t=120$с}
	\label{fig:120}
\end{figure}
\begin{figure}[H]
	\center
	\includegraphics[width=.6\linewidth]{imgs_11/600s.png}
	\caption{Распределение температуры в $t=600$с}
	\label{fig:600}
\end{figure}

\begin{figure}[H]
	\center
	\includegraphics[width=.6\linewidth]{imgs_11/temp.png}
	\caption{Зависимость температуры в верхней части стакана кофе от времени ($x=2 , y= 0.97$ см).}
	\label{fig:temp}
\end{figure}

\newpage
\section{Исходный код}
\lstinputlisting[label={lst:task11}, caption={Исходный код задания}]{task_11.py}


\end{document}
