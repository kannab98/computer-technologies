\input{text/pre}

\NewDocumentCommand{\codeword}{v}{%
\texttt{\textcolor{gray}{#1}}%
}
\input{python.tex}

\title{Компьютерные технологии}
\author{Горев С.}



\begin{document}
\maketitle

\textbf{Задание 8} Рассчитайте эффективную траекторию ракеты, предназначенной для наименее
затратного по топливу запуска с Земли искусственного спутника Марса. Постройте зависимости
скорости и координаты ракеты от времени, а также оцените точность интегрирования в
зависимости от схемы интегрирования и величины шага интегрирования

\section{Подход к решению задачи}
Оптимальной по топливу траекторией перехода с одной орбиты на другую является
Гомановская траектория - в ходе нее ракета выполняет два импульса - начальный $\Delta V_1$,
для достижения конечной орбиты, и конечный $\Delta V_2$, для выхода на конечную орбиту.
В данном решении будет использоваться Гомановская траектория полета - т.е. всего будет два импульса приращения скорости.

Теоретические значения приращения скорости
\begin{equation}
	\Delta V_1 = V_E \left(\frac{V_E}{V_p}-1\right), \quad \Delta V_2 = V_M \left(1- \frac{V_M}{V_p}\right),
\end{equation}
где $V_E, V_M$ - орбитальные скорости Земли и Марса, $V_p = \sqrt{\frac{V_E^2 + V_M^2}{2}}$. Однако, данные значения справедливы
только когда не учитывается гравитационное притяжение тел (Земли, Марса), поэтому при учете этих сил
траектория будет искажена. В данной работе значение первого импульса будет находится путем оптимизации, условием которой будет
достижение геостационарной орбиты (ГСО) Марса, при минимальной радиальной скорости ракеты при сближении.

\section{Описание системы тел}
В нашем случае система состоит из Солнца, Земли, Марса и ракеты. Орбиты планет будет считать круговыми, с фиксированными
радиусами и орбитальными скоростями. Солнце неподвижно и расположено в начале координат.

Ракета будет запускаться с ГСО Земли, при этом будет расположена от Солнца дальше, чем Земля. Таким образом, начальные
параметры ракеты можно записать как 
\begin{equation}
		x(0) = R_E + GSO_E, y(0) = 0, V_x(0)=0, V_y(0) = V_E + V_{GSO, E} + \Delta V_1
\end{equation}
, где $R_E=1.5 \cdot 10^{11}$ м - радиус орбиты Земли, $GSO_E = 6.371 \cdot 10^6 + 35.786 \cdot 10^6 = 42.157 \cdot 10^6$ м -
величина радиуса ГСО Земли, при отсчете от центра Земли, $V_{GSO, E} = 3065$ м/с - скорость на ГСО Земли.

Важным параметром является момент, в который ракета производит запуск с ГСО Земли по отношению к положению Марса.
В данной работе оптимальное положение Марса будет находится путем оптимизации тем же методом, что и нахождение оптимального
начального импульса.

Таким образом, в задаче будут следующие ключевые этапы:
\begin{enumerate}
	\item Ракета начинает движение с ГСО Земли. Движение ракеты описывается всемирным законом тяготения.
	\item Определяется оптимальный начальный импульс, путем минимизации некоторой функции $M=M(\Delta V_1, \theta_M(0))$, 
	где $\theta_M(0)$ - начальный угол положения Марса в полярных координатах.
	\item Найдя оптимальные значения $\Delta V_1, \theta_M(0)$, рассчитывается полет ракеты до Марса, до момента максимального
	сближения.
	\item Дальше ракете придается дополнительный импульс $\Delta V_2$, после которого ракета выходит на орбиту Марса.
\end{enumerate}


\section{Описание движения ракеты в системе}
Для моделирования влияния нескольких тел на движение ракеты, воспользуемся вторым законом Ньютона и законом
всемирного тяготения:
\begin{equation}
	\vec{F} = m\vec{a} = \vec{F_g} = G m \sum_{i=0}^{N-1}\frac{m_i}{R_i^2}\vec{e_i},
\end{equation}
где $i$ - индекс, $m$ - масса ракеты, $m_i$ - масса $i$-ого тела, $R_i$ - расстояние между
ракетой и $i$-м телом, $\vec{e_i}$ - единичный вектор, направленный от ракеты к телу с индексом $i$.
Влияние ракеты на движение планет и Солнца не учитывается.

Учтем, что $\vec{r''}=\vec{a}, \vec{r'} = \vec{v}$, тогда
\begin{equation}
	\vec{v'} = G \sum_{i=0}^{N-1}m_i\frac{\vec{r_i}-\vec{r}}{R_i^3}, \quad
	\vec{r'} = \vec{v}
\end{equation}
где $\vec{r_i}$ - радиус вектор положения тела с индексом $i$.
Решением полученной системы уравнений будет траектория полета ракеты $\vec{r}(t), \vec{v}(t)$.
Именно эта системы и будет моделироваться в программе.

\section{Определение оптимальных параметров}
Ранее была введена функция $M=M(\Delta V_1, \theta_M(0))$, минимизацией значения которой будут найдены оптимальные 
параметры $\Delta V_1, \theta_M(0)$. Опишем критерии, по которым будет определяться нахождение оптимальных параметров.
\begin{enumerate}
	\item Критерий достижения ГСО Марса. Минимальное расстояние до Марса должно быть меньше или равно величине
	ГСО Марса (считая от центра координаты Марса)
	\item Минимальная радиальная скорость. Радиальная скорость ракеты при минимальном расстоянии до Марса должна быть минимальной, в идеале 0.
	\item Оптимальная траектория полета. Максимальное значение радиуса орбиты ракеты должно быть близко к значению орбиты Марса.
\end{enumerate}
Исходя из поставленных критериев была составлена функция, возвращаемое значение которой является
\begin{equation}
	M=M(\Delta V_1, \theta_M(0)) = \abs{d2M_{min}} + V_{r min}^2 + \abs{d2MO},
\end{equation}
где $d2M_{min}$ - расстояние до ГСО Марса при максимальном сближении, $V_{r min}$ - радиальная скорость при максимальном сближении,
$d2MO$ - разность между максимальным радиусом орбиты ракеты и орбитой Марса.
Минимизируя данную величину, будут найдены оптимальные параметры $\Delta V_1, \theta_M(0)$.

\section{Результаты моделирования}
Моделирование системы производится на языке Python.
Используется библиотека SciPy и метод \text{integrate.solve\_ivp}.
В методе \text{integrate.solve\_ivp} использует алгоритм Рунге-Кутты 5-го порядка ('Radau'). Шаг интегрирования выбирается так,
чтобы ошибка не превышала наперед заданного значения относительной $\varepsilon_r \leq 10^{-3}$ и абсолютной
ошибок $\varepsilon_a \leq 10^{-6}$ интегрирования.

Исходный код приведен в листинге \ref{lst:task9}.

Результаты моделирования приведены на рисунках \ref{fig:trj}, \ref{fig:crds}, \ref{fig:vel}.

Изначальные значения $\Delta V_1=2400 m/s, \theta_M(0)=0.91 rads$. Начальные координаты и скорости ракеты
\begin{equation}
	x(0) = 1.5 \cdot 10^{11} + 42.157 \cdot 10^6, y(0) = 0, V_x(0)=0, V_y(0) = 29783 + 3065 + \Delta V_1
\end{equation}

Оптимальные значения $\Delta V_1, \theta_M(0)$, найденные в ходе оптимизации
\begin{equation}
	\Delta V_1^{opt} = 2327.05 m/s \quad \theta_M(0)^{opt} = 0.925 rads,
\end{equation}
для сравнения, теоретической значение $\Delta V_1=2943.45$ м/с, однако из-за влияния гравитации Земли
оптимальное значение отличается от теоретического.

Момент достижения Марса отмечен на графиках вертикальной линией, и составляет $t_{ETA} = 261.13$ дней с момента запуска.
В момент достижения Марса сближение с центром Марса составляет $10475$ км, при величине ГСО Марса $20389.5$ км. Радиальная скорость
при максимальном сближении составляет $-258$ м/с.

После достижения Марса, ракете придается второй импульс $\Delta V_2 = 2647.9$ м/с, что является теоретическим значением.
После этого, ракета выходит на орбиту Марса, что можно пронаблюдать на графике скорости на рис. \ref{fig:vel} - после момента достижения Марса,
скорость начинает сильно осциллировать, что характерно при выходе на орбиту планеты. Также приведен график координаты в увеличенном мастшатбу на рис. 
\ref{fig:orb}, на котором наблюдается осцилляция вокруг координаты Марса.

\begin{figure}
	\center
	\includegraphics[width=.8\linewidth]{imgs_9/trj.png}
	\caption{Траектории движений}
	\label{fig:trj}
\end{figure}

\begin{figure}
	\center
	\includegraphics[width=.99\linewidth]{imgs_9/crds.png}
	\caption{Координаты от времени}
	\label{fig:crds}
\end{figure}

\begin{figure}
	\center
	\includegraphics[width=.99\linewidth]{imgs_9/vel.png}
	\caption{Скорость ракеты от времени}
	\label{fig:vel}
\end{figure}

\begin{figure}
	\center
	\includegraphics[width=.99\linewidth]{imgs_9/orb.png}
	\caption{Выход ракеты на орбиту Марса - координата ракеты осциллирует вокруг координаты Марса}
	\label{fig:orb}
\end{figure}

\newpage
\section{Исходный код}
\lstinputlisting[label={lst:task9}, caption={Исходный код задания}]{task_9.py}


\end{document}