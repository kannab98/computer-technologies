\input{text/pre}

\NewDocumentCommand{\codeword}{v}{%
\texttt{\textcolor{gray}{#1}}%
}
\input{python.tex}

\title{Компьютерные технологии}
\author{Гвоздков Е.}



\begin{document}
\maketitle

\textbf{Задание 8} Постройте модель Солнечной системы. Рассчитайте параметры траектории кометы,
попавшей в Солнечную систему извне. Постройте зависимости скорости и координаты кометы
от времени при различных начальных параметрах, а также оцените точность интегрирования в
зависимости от схемы интегрирования и величины шага интегрирования.

\section{Описание модели Солнечной системы}
Для описания движения планет и кометы в поле тяготения Солнца примем несколько приближений:
\begin{enumerate}
	\item Планеты не влияют гравитацией друг на друга
	\item Описание движения будет происходить в плоскости, т.е. не учитывается координата $z$
	\item Комета не влияет на орбиты планет Солнечной системы
	\item Солнце неподвижно в начале координат
\end{enumerate}
Поскольку влиянием планет друг на друга принебрегается, их орбиты описываются определенным образом.
Траектория орбиты представляет из себя эллипс, в фокусе которого расположено тяготеющее тело, в данном случае Солнце.

\subsection{Описание движения планет}
Уравнение эллипса орбиты в полярных координатах задается следующим образом:
\begin{equation}
	r = \frac{a(1-e^2)}{1 + e \cos(\theta + \alpha)},
\end{equation}
где $a$ - большая полуось эллипса, $e$ - эксцентриситет, $\alpha$ - угловой сдвиг
эллипса относительно $\theta=0$.

Закон невозмущенного движения тела по эллиптической орбите из второго закона Кеплера имеет вид
\begin{equation}
	r^2\frac{d\theta}{d t} = \const = \sqrt{\mu a (1-e^2)},
\end{equation}
где $\mu = GM$ - гравитационный параметр ($G$ - гравитационная постоянная, $M$ - масса Солнца).

% Таким образом, для полного описания орбиты небесного тела необходимы следующие параметры
\begin{figure}[h!]
	\centering
	\includegraphics[width=0.8\linewidth]{imgs_8/solar.png}
	\caption{Модель Солнечной системы}
    \label{fig:system}
\end{figure}


\section{Описание движения кометы в Солнечной системе}
Движение тела в поле тяготения описываются законом всемирного тяготения
\begin{equation}
	F_g = G\frac{m_1 m_2}{R^2},
\end{equation}
где $m_1, m_2$ - массы тел, $R$ - расстояние между телами. Сила при этом направлена от кометы к планете.
В случае нескольких тел $N$, действующих
гравитацией на конкретное тело (комету), силы суммируются, и закон примет виде

\begin{equation}
	F_g = G m \sum_{i=0}^{N-1}\frac{m_i}{R_i^2},
\end{equation}
где $i$ - индекс, $m$ - масса кометы, $m_i$ - масса $i$-ой планеты, $R_i$ - расстояние между
кометой и $i$-той планетой.

Для моделирования влияния нескольких тел на движение кометы, воспользуемся вторым законом Ньютона:
\begin{equation}
	\vec{F} = m\vec{a} = \vec{F_g} = G m \sum_{i=0}^{N-1}\frac{m_i}{R_i^2}\vec{e_i},
\end{equation}
где $\vec{e_i}$ - единичный вектор, направленный от кометы к планете с индексом $i$.
Приведем выражение выше в другом виде
\begin{equation}
    \vec{r''} = G \sum_{i=0}^{N-1}m_i\frac{\vec{r_i}-\vec{r}}{R_i^3},
\end{equation}
где $\vec{r_i}$ - радиус вектор положения планеты с индексом $i$. Введем $\vec{v} = \vec{r'}$,
тогда получим следующую систему уравнений
\begin{equation}
	\vec{v'} = G \sum_{i=0}^{N-1}m_i\frac{\vec{r_i}-\vec{r}}{R_i^3}, \quad
	\vec{r'} = \vec{v}
\end{equation}

\section{Результаты моделирования}
Описание и моделирование системы производится на языке Python.
Начальными параметрами моделирования выступают первоначальные положения планет Солнечной системы,
а также начальные координаты и скорость кометы.

Планеты и их движение описываются классом $\codeword{CelestialBody}$ в файле $\codeword{SolarSystem.py}$.
Каждая планета - инстанция класса. Комета описывается отдельным классом $\codeword{Comet}$, в котором также 
присутствует метод \codeword{evaluate_model}, который является основным в моделировании.

\newpage
\section{Исходный код}
\lstinputlisting[label={lst:task8}, caption={Исходный код задания}]{SolarSystem.py}
\lstinputlisting[label={lst:task8_2}, caption={Исходный код задания}]{simulation.py}


\end{document}