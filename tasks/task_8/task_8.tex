\input{text/pre}

\NewDocumentCommand{\cdw}{v}{\textsf{\textcolor[HTML]{2489F5}{#1}}}
\input{python.tex}

\title{Компьютерные технологии}
\author{Гвоздков Е.}



\begin{document}
\maketitle

\textbf{Задание 8} Постройте модель Солнечной системы. Рассчитайте параметры траектории кометы,
попавшей в Солнечную систему извне. Постройте зависимости скорости и координаты кометы
от времени при различных начальных параметрах, а также оцените точность интегрирования в
зависимости от схемы интегрирования и величины шага интегрирования.

\section{Описание модели Солнечной системы}
Для описания движения планет и кометы в поле тяготения Солнца примем несколько приближений:
\begin{enumerate}
	\item Планеты не влияют гравитацией друг на друга
	\item Описание движения будет происходить в плоскости, т.е. не учитывается координата $z$
	\item Комета не влияет на орбиты планет Солнечной системы
	\item Солнце неподвижно в начале координат
\end{enumerate}
Поскольку влиянием планет друг на друга принебрегается, их орбиты описываются определенным образом.
Траектория орбиты представляет из себя эллипс, в фокусе которого расположено тяготеющее тело, в данном случае Солнце.

Уравнение эллипса орбиты в полярных координатах задается следующим образом:
\begin{equation}
	r = \frac{a(1-e^2)}{1 + e \cos(\theta + \alpha)},
\end{equation}
где $a$ - большая полуось эллипса, $e$ - эксцентриситет, $\alpha$ - угловой сдвиг
эллипса относительно $\theta=0$.

Закон невозмущенного движения тела по эллиптической орбите из второго закона Кеплера имеет вид
\begin{equation}
	r^2\frac{d\theta}{d t} = \const = \sqrt{\mu a (1-e^2)},
\end{equation}
где $\mu = GM$ - гравитационный параметр ($G$ - гравитационная постоянная, $M$ - масса Солнца).

% Таким образом, для полного описания орбиты небесного тела необходимы следующие параметры
\begin{figure}[h!]
	\centering
	\includegraphics[width=0.5\linewidth]{imgs_8/solar.png}
	\caption{Модель Солнечной системы}
    \label{fig:system}
\end{figure}


\section{Описание движения кометы в Солнечной системе}
Движение тела в поле тяготения описываются законом всемирного тяготения
\begin{equation}
	F_g = G\frac{m_1 m_2}{R^2},
\end{equation}
где $m_1, m_2$ - массы тел, $R$ - расстояние между телами. Сила при этом направлена от кометы к планете.
В случае нескольких тел $N$, действующих
гравитацией на конкретное тело (комету), силы суммируются, и закон примет виде

\begin{equation}
	F_g = G m \sum_{i=0}^{N-1}\frac{m_i}{R_i^2},
\end{equation}
где $i$ - индекс, $m$ - масса кометы, $m_i$ - масса $i$-ой планеты, $R_i$ - расстояние между
кометой и $i$-той планетой.

Для моделирования влияния нескольких тел на движение кометы, воспользуемся вторым законом Ньютона:
\begin{equation}
	\vec{F} = m\vec{a} = \vec{F_g} = G m \sum_{i=0}^{N-1}\frac{m_i}{R_i^2}\vec{e_i},
\end{equation}
где $\vec{e_i}$ - единичный вектор, направленный от кометы к планете с индексом $i$.
Приведем выражение выше в другом виде
\begin{equation}
    \vec{r''} = G \sum_{i=0}^{N-1}m_i\frac{\vec{r_i}-\vec{r}}{R_i^3},
\end{equation}
где $\vec{r_i}$ - радиус вектор положения планеты с индексом $i$. Введем $\vec{v} = \vec{r'}$,
тогда получим следующую систему уравнений
\begin{equation}
	\vec{v'} = G \sum_{i=0}^{N-1}m_i\frac{\vec{r_i}-\vec{r}}{R_i^3}, \quad
	\vec{r'} = \vec{v}
\end{equation}

\section{Результаты моделирования}
Описание и моделирование системы производится на языке Python.
Начальными параметрами моделирования выступают первоначальные положения планет Солнечной системы,
а также начальные координаты и скорость кометы. Параметры орбит планет считаются фиксированными и заданы предварительно
на основе информации из открытых источников.

Планеты и законы их движения описываются классом $\cdw{CelestialBody}$ в файле $\cdw{SolarSystem.py}$.
Каждая планета - инстанция класса. Комета описывается отдельным классом $\cdw{Comet}$, в котором также 
присутствует метод класса \cdw{evaluate_model}, который является основным в моделировании.

В качетсве параметров кометы выступают масса, начальные координаты и скорости. Также нужно задать время моделирования
\cdw{t_end} - до какого момента времени в секундах моделировать движение. Также есть возможность выбрать метод
интгрирования, задав соответствующее строковое значение переменной \cdw{integration_method}
\begin{lstlisting}[caption={Блок параметров симуляции},captionpos=b]
	TheComet = Comet("Comet",
                         8*10**22,    # Масса, кг
                         1.5*10**12,  # Начальный радиус, м
                         0.8*np.pi,   # Начальный угол, радианы
                         0,           # Начальная радиальная скорость, м/c
                         10000        # Начальная угловая скорость, м/с
				         )
	integration_method = 'Radau'
	t_end = 20*10**8
\end{lstlisting}
\subsection{Скорость и координата кометы в зависимости от начальных параметров}
Начальные параметры в тексте будем указывать как $m$ - масса кометы, $R_0$ - начальный радиус
траектории кометы, $\theta_0$ - начальный угол, $v_r$ - начальная радиальная скорость,
$v_\theta$ - начальная угловая скорость. В основном будет метод интегрирования "Radau" -
метод Рунге-Кутты 5-го порядка. параметры моделирования будут указыаваться следующим образом
$[m, R_0, \theta_0, v_r, v_\theta]$
\begin{lstlisting}
	"ConfigName": [8*10**22, 1.5*10**12, 0, 0, 10000]
\end{lstlisting}


\subsubsection*{Модель комета - Солнце}
Рассмотрим простой случай, когда в модели отсутствуют другие планеты. Начальные параметры кометы:
\begin{lstlisting}
	"SunComet": [8*10**22, 1.5*10**12, 0, 0, 10000]
\end{lstlisting}

\begin{figure}[H]
    \centering
    \begin{subfigure}{0.49\linewidth}
        \centering
		\includegraphics[width=1\linewidth]{imgs_8/rSunComet.png}
		\caption{Координата $r(t)$}
	\end{subfigure}
	\begin{subfigure}{0.49\linewidth}
        \centering
		\includegraphics[width=1\linewidth]{imgs_8/vSunComet.png}
		\caption{Скорости $V_r(t), V_\theta(t)$}
	\end{subfigure}
	\begin{subfigure}{0.49\linewidth}
        \centering
		\includegraphics[width=1\linewidth]{imgs_8/trjSunComet.png}
		\caption{Траектория}
    \end{subfigure}
    \caption{Модель комета - Солнце.}
    \label{fig:SunComet}
\end{figure}
На рис. \ref{fig:SunComet}(c) представлена траектория кометы - в данном случае, т.к. других планет нету, то комета
невозмущенно вращается вокруг Солнца по эллипсу.
\subsubsection*{Комета появляется внутри Солнечной системы}
Если же мы, не меняя начальных параметров, "включим" влияние планет, картина сильно изменится:

\begin{figure}[H]
    \centering
    \begin{subfigure}{0.49\linewidth}
        \centering
		\includegraphics[width=1\linewidth]{imgs_8/r.png}
		\caption{Координата $r(t)$, а также координаты планет $r_i(t)$}
	\end{subfigure}
	\begin{subfigure}{0.49\linewidth}
        \centering
		\includegraphics[width=1\linewidth]{imgs_8/v.png}
		\caption{Скорости $V_r(t), V_\theta(t)$}
	\end{subfigure}
	\begin{subfigure}{0.49\linewidth}
        \centering
		\includegraphics[width=1\linewidth]{imgs_8/trj.png}
		\caption{Траектория кометы и планет}
    \end{subfigure}
    \caption{Результаты моделирования при "включении" влияния планет}
    \label{fig:default}
\end{figure}
Траектория кометы становится сильно искаженной (см. рис. \ref{fig:default}), т.к. она подвергается сильному влиянию Сатурна и Юпитера.

\subsubsection*{Комета прилетает извне Солнечной системы}
Рассмотрим случай когда комета прилетает извне Солнечной системы, и проходит сквозь нее. Начальные параметры кометы:
\begin{lstlisting}
	"Outside": [8*10**22, 10**13, 0, -2000, 1000]
\end{lstlisting}

\begin{figure}[H]
    \centering
    \begin{subfigure}{0.49\linewidth}
        \centering
		\includegraphics[width=1\linewidth]{imgs_8/rOutside.png}
		\caption{Координата $r(t)$, а также координаты планет $r_i(t)$}
	\end{subfigure}
	\begin{subfigure}{0.49\linewidth}
        \centering
		\includegraphics[width=1\linewidth]{imgs_8/vOutside.png}
		\caption{Скорости $V_r(t), V_\theta(t)$}
	\end{subfigure}
	\begin{subfigure}{0.49\linewidth}
        \centering
		\includegraphics[width=1\linewidth]{imgs_8/trjOutside.png}
		\caption{Траектория}
    \end{subfigure}
    \caption{Комета прилетает извне Солнечной системы}
    \label{fig:Outside}
\end{figure}

Как можно видеть (см. рис. \ref{fig:Outside}), комета была "поймана" Юпитером, а затем отправлена за пределы системы.

\subsubsection*{Комета прилетает извне Солнечной системы с малым воздействием}
Похожий сценарий может иметь совершенно другой исход.
Например, планеты практически не подействуют на комету. Начальные параметры кометы:
\begin{lstlisting}
	"RunBy":[8*10**22, 10**13, 3.14, -9000, -1500]
\end{lstlisting}

\begin{figure}[H]
    \centering
    \begin{subfigure}{0.49\linewidth}
        \centering
		\includegraphics[width=1\linewidth]{imgs_8/rRunBy.png}
		\caption{Координата $r(t)$, а также координаты планет $r_i(t)$}
	\end{subfigure}
	\begin{subfigure}{0.49\linewidth}
        \centering
		\includegraphics[width=1\linewidth]{imgs_8/vRunBy.png}
		\caption{Скорости $V_r(t), V_\theta(t)$}
	\end{subfigure}
	\begin{subfigure}{0.49\linewidth}
        \centering
		\includegraphics[width=1\linewidth]{imgs_8/trjRunBy.png}
		\caption{Траектория}
    \end{subfigure}
	\caption{Комета прилетает извне Солнечной системы, и практически
	не взаимодуйствует с планетами}
    \label{fig:RunBy}
\end{figure}
Здесь (см. рис. \ref{fig:RunBy}) можно наблюдать близкую к гиперболической траекторию кометы -
двигаясь по такой траектории тело сможет удалиться на бесконечность.
\subsection{Оценка точности интегрирования}

\section{Выводы}
В работе была имплементирована модель Солнечной систсемы, смоделировано движение кометы
и воздействие на ее траекторию гравитации планет и Солнца. 

При малоразличных начальных параметрах можно получить совершенно разыне картины траекторий,
т.е. система является хаотичной.


\newpage
\section{Исходный код}
\lstinputlisting[label={lst:task8}, caption={SolarSystem.py}]{SolarSystem.py}
\lstinputlisting[label={lst:task8_2}, caption={simulation.py}]{simulation.py}
\lstinputlisting[label={lst:task8_3}, caption={configs.py}]{configs.py}


\end{document}