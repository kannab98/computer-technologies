\documentclass[a4paper,14pt]{extarticle}

\usepackage[utf8]{inputenc}
\usepackage[T2A]{fontenc}
\usepackage[english, russian]{babel}

\usepackage[mode=buildnew]{standalone}
\usepackage{setspace}


% Различные пакеты
\usepackage{
	amssymb, amsfonts, amsmath, amsthm, physics,
	cancel, indentfirst,makecell,multirow, 
	graphicx, tikz, mathtools, float, setspace,caption,subcaption
} 

\usepackage{mathtools}

% Эта опция включает нумерацию только у тех формул,
% на которые есть ссылка в документе
\mathtoolsset{showonlyrefs=true} 

 % Цвета для гиперссылок
\definecolor{linkcolor}{HTML}{000000} % цвет ссылок
\definecolor{urlcolor}{HTML}{799B03} % цвет гиперссылок
 
\usepackage{xcolor}
\usepackage[
    unicode, 
    colorlinks, 
    urlcolor=urlcolor, 
    linkcolor=linkcolor,
    citecolor=linkcolor
]{hyperref}

% Увеличенный межстрочный интервал, французские пробелы
\linespread{1.2} 
\frenchspacing 

%%%%%%%%%%%%%%%%%%%%%%%%%%%%%%
%  Пользовательские команды  %
%%%%%%%%%%%%%%%%%%%%%%%%%%%%%%


\makeatletter
    \newcommand{\fftStar}[1]{\mathfrak{F}^*\qty[#1]}
    \newcommand{\fftNoStar}[1]{\mathfrak{F}\qty[#1]}
    \newcommand{\fft}{
                 \@ifstar
                 \fftStar%
                 \fftNoStar%
    }
\makeatother

\makeatletter
    \newcommand{\ifftNoStar}[1]{\mathfrak{F}^{-1}\qty[#1]}
    \newcommand{\ifftStar}[1]{\qty(\mathfrak{F}^{-1}\qty[#1])^*}
    \newcommand{\ifft}{
                 \@ifstar
                 \ifftStar%
                 \ifftNoStar%
    }
\makeatother

\newcommand{\mean}[1]{\langle#1\rangle}
\newcommand\ct[1]{\text{\rmfamily\upshape #1}}
\newcommand*{\const}{\ct{const}}
\renewcommand{\phi}{\varphi}
\renewcommand{\epsilon}{\varepsilon}
%\renewcommand{\sigma}{\varsigma}


\captionsetup{subrefformat=parens}

\usepackage{array}
\usepackage{pstool}


% Диагональная ячейка в таблице ( типа |a/b|)
\newcolumntype{x}[1]{>{\centering\arraybackslash}p{#1}}
\newcommand\diag[4]{%
  \multicolumn{1}{p{#2}|}{\hskip-\tabcolsep
  $\vcenter{\begin{tikzpicture}[baseline=0,anchor=south west,inner sep=#1]
      \path[use as bounding box] (0,0) rectangle (#2+2\tabcolsep,\baselineskip);
      \node[minimum width={#2+2\tabcolsep},minimum height=\baselineskip+\extrarowheight] (box) {};
      \draw (box.north west) -- (box.south east);
      \draw (box.south west) -- (box.north west);
      \node[anchor=south west] at (box.south west) {\footnotesize#3};
      \node[anchor=north east] at (box.north east) {\footnotesize#4};
 \end{tikzpicture}}$\hskip-\tabcolsep}}

%%%%%%%%%%%%%%%%%%%%%%%%%%%%%
%  Геометрия и колонтитулы  %
%%%%%%%%%%%%%%%%%%%%%%%%%%%%%


\usepackage{geometry}
\geometry       
    {
        left            =   2cm,
        right           =   2cm,
        top             =   2.5cm,
        bottom          =   2.5cm,
        bindingoffset   =   0cm
    }

% Настройка содержания, точки после нумераций
\usepackage{tocloft}
\addto\captionsrussian{\renewcommand{\contentsname}{Оглавление}}
\renewcommand{\cftsecleader}{
	\cftdotfill{\cftdotsep}}
% \renewcommand{\thesection}{
	% \arabic{section}.}
% \renewcommand{\thesubsection}{
	% \arabic{section}.\arabic{subsection}.}
% \renewcommand{\thesubsubsection}{
	% \arabic{section}.\arabic{subsection}.\arabic{subsubsection}.}     
\usepackage[explicit]{titlesec}

% Колонтитулы
\usepackage{fancyhdr} 
	\pagestyle{plain} 
	\fancyhead{} 
	\fancyhead[R]{} 
	\fancyhead[L]{} 
	\fancyfoot{} 
	\fancyfoot[C]{\thepage} 
